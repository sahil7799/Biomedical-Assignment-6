\documentclass{article}
% Comment the following line to NOT allow the usage of umlauts
\usepackage[utf8]{inputenc}
% Uncomment the following line to allow the usage of graphics (.png, .jpg)
%\usepackage{graphicx}

% Start the document
\begin{document}

% Create a new 1st level heading
\section{WHAT DO BIOMEDICAL ENGINEERS DO}

A Biomedical Engineer's job include designing biomedical equipment and devices to help people recover or enhance their health.
Internal devices, such as stents or artificial organs, as well as external devices, such as braces and supports, are examples.
It may also entail the design and adaptation of medical equipment.
It's a job that necessitates a strong understanding of computing, biology, and engineering, as well as a creative mindset and problem-solving abilities.

\section{TREATMENT MECHANISMS}

\subsection{OXYGEN}

The delivery of additional oxygen using a nasal cannula or a more intrusive face mask is usually the primary form of treatment for mild respiratory insufficiency.
The oxygen is usually delivered in cylinders, which are either tiny for transportation or big for fixed patients and longer-term supplies.


Although oxygen concentrators are an appealing option to tanks, they are rarely used in hospital settings for caring for COVID-19 patients.
Oxygen concentrators take oxygen from the air and deliver it to the patient on demand.
Concentrators are available in a variety of sizes, ranging from a small portable shoulder bag to larger fixed units for patients who require oxygen 24 hours a day.
High flow nasal oxygen (HFNO) is a type of oxygen delivery that delivers warmed and humidified oxygen at high flow rates (usually tens of litres/min) at body temperature and up to 100 percent RH and 100 percent oxygen to avoid drying out the airways.

\subsection{VENTILATORS}

Patients who are unable to breathe on their own must be placed on a ventilator.
Patients in an advanced stage of respiratory distress are frequently intubated and sedated at the start of treatment since ventilators can replace breath function.


Patients in an advanced stage of respiratory distress are frequently intubated and sedated at the start of treatment since ventilators can replace breath function.
They are complicated devices that give healthcare providers a lot of flexibility in terms of adjusting assisted breathing settings and eventually weaning healing patients off the ventilator.
Modern ventilators are usually pressure-controlled closed loops that can detect spontaneous breathing and provide synchronised aid to recovering patients.
They also allow the patient to alter the composition of the gas he or she breathes, ranging from regular air to 100 percent oxygen. They normally get their supply from the hospital's gas supply network, but they can also be connected to oxygen tanks or oxygen concentrators if there isn't one.

\subsection{CONTINUOUS POSITIVE AIRWAY PRESSURE (CPAP)}

Continuous Positive Airway Pressure (CPAP) is the next step in treating COVID-19 patients. CPAP was originally designed to avoid airways collapse in sleep apnoea patients, but it has been demonstrated to be beneficial to COVID patients if used early enough in the disease's course.


A well-fitting face mask is an important part of a CPAP machine, but it may be rather bothersome.
Because CPAP effectively resists some resistance to expiration, it is only suited for patients who are capable of some breathing strength.
There are variants that adapt the level of pressure automatically to the patient's breathing characteristics (APAP) or have distinct levels of pressure for inspiration and expiration (BiPAP). CPAP normally provides the patient with (filtered) air, but most masks feature a port for adding oxygen to the mix (BiPAP).

\subsection{PATIENT MONITORING}

The monitoring equipment, which keeps track of some of the patient's vitals, especially when they are ventilated and sedated, but also during their recovery phase to ensure the ventilation regime is optimised for their condition, is an important part of the ICU equipment.
Ventilators already have their own set of patient parameters, but patient monitors are usually distinct devices because they are still relevant when the patient can breathe on their own.
The amount of oxygen in a patient's bloodstream (SpO2), which is evaluated by pulse oximetry, which uses optics within a finger clamp, is one of the most important metrics for COVID-19 patients.
Pulse oximetry is often used for the duration of a patient's stay in the intensive care unit.
Modern patient monitors provide a plethora of additional patient parameters, all the way down to breathing waveforms, allowing doctors to fine-tune their patient treatment.

\subsection{ECG AND ULTRASOUND MACHINE}

On two fronts, doctors around the world are battling the COVID-19 pandemic.
The virus is still primarily known as a lung infection; the predominant symptom is a dry cough, which can lead to viral pneumonia in severe instances.
However, the new coronavirus also attacks other key organs, including the heart, which is located between the lungs.


The increased threat means that hospitals' anti-virus arsenals must include more than just mechanical ventilators, which assist patients in breathing.
They will also rely on electrocardiogram (ECG) and ultrasonography, two technologies that have been standard on medical wards for decades.
But this isn't a tale about hospitals relying on two tried-and-true warhorses.
Today's ECG and ultrasound machines come equipped with cutting-edge software that can track COVID-19's attack on the heart, allowing clinicians to intervene early and save lives.


The virus's adaptability was an unpleasant surprise.
In March, Chinese researchers published a study of 416 individuals hospitalised with the coronavirus that revealed nearly one-fifth (19 percent) of them suffered heart damage.
Over half (51 percent) of these patients, who were on average 64 years old, died as a result of the cardiac injury.
It should come as no surprise that patients with pre-existing cardiovascular illness were more sensitive to COVID-19-induced cardiac damage in the study.

% Uncomment the following two lines if you want to have a bibliography
%\bibliographystyle{alpha}
%\bibliography{document}

\end{document}